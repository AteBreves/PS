%
% Capítulo 1
%
\chapter{Introdução} \label{cap:intro}

A vasta maioria das casas, lojas e escritórios recorrem a serviços de abastecimento de água. O custo deste serviço é calculado, normalmente, através de uma estimativa da quantidade de água gasta (por norma mensalmente) e, periodicamente, um funcionário da empresa que fornece o serviço tem de se deslocar à localização do contador para que seja verificado o consumo real de água para o acerto do pagamento. 
O crescimento tecnológico contemporâneo facilitou e incentivou o acesso de grande parte da população aos vários equipamentos e plataformas tecnológicas que nos permitem efetuar diversas tarefas que outrora necessitariam de outra burocracia ou até serviço presencial. A entrega da leitura da contagem da água é algo que pode ser modernizado e automatizado, porém reconhecemos que soluções que envolvam, por exemplo, a substituição dos equipamentos contadores, possam acarretar um custo logístico e financeiro não justificável para a empresa. 
Neste projeto, propomos uma solução que permite modernizar o processo de entrega de leituras de água apenas envolvendo a implementação de um sistema informático que permite aos clientes enviar fotografias dos seus contadores que depois o sistema processará para obter o valor da contagem.


%
% Secção 1.1
%
\section{Objetivo do Projeto} \label{sec11}

De forma a não ser necessário fazer acertos de pagamentos e permitir que o cliente pague realmente o valor que consumiu, ao invés do consumo estimado, este projeto tem como propósito o desenvolvimento de um sistema informático composto por, de entre outros elementos, um elemento que o cliente utiliza para comunicar ao fornecedor o seu consumo de água. Este poderá também ser utilizado para indicar estatísticas de consumo e notficar o cliente de informações pertinentes relativas a este serviço. Para além deste elemento, também vai ser realizado um servidor cuja função principal é comunicar com o elemento dos utilizadores e interagir com o local onde estão guardadas as informações dos clientes.

\section{Organização do Documento} \label{sec12}
O restante relatório encontra-se organizado em quatro capítulos. No capítulo \ref{cap:trabrelacionado} vamos avaliar e debater soluções já existentes no mercado cuja função se aproxima da deste projeto, bem como os vários equipamentos e conceitos utilizados na área cujo projeto se insere. No capítulo \ref{cap:analise} estudaremos os vários problemas do projeto, detalhando os vários requisitos que o sistema terá de cumprir para satisfazer o seu propósito. No capítulo \ref{cap:abordagem} vamos analisar as várias abordagens aos problemas do projeto. No capítulo \ref{cap:implementacao} vamos analisar as várias escolhas e decisões que foram efetuadas no desenvolvimento deste projeto.
%O capítulo \ref{cap:trabrelacionado} será onde vamos comparar e explicar as várias estratégias de abordagem às soluções apresentadas.
% Por fim, o capítulo 5 (Aspetos de Implementação) será onde vamos analisar diretamente o código fonte da solução e explicar os diversos detalhes e escolhas efetuadas durante a escrita do código.