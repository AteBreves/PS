\chapter{Implementação do Sistema} \label{cap:implementacao}

Este sistema, por ser desenvolvido na plataforma Outsystems, é constituído por módulos, que representam os vários elementos do sistema.\\
O módulo onde será desenvolvido o servidor e a interface para os utilizadores é o módulo WaterWatcher. O módulo que servirá para manter os dados será o WaterWatcherService.\\
Por fim desenvolvemos também um módulo de testes (WaterWatcherTests) e um módulo que simula as interações com o sistema informático da empresa fornecedora de água, denominado SimulCompany.
Este capítulo tem como propósito clarificar e justificar as várias decisões que tomámos no desenvolvimento dos vários módulos do projeto.\\
Na secção \ref{modww} vamos abordar o módulo WaterWatcher, na secção \ref{modwws} o módulo WaterWatcherService e na secção \ref{modsc} o módulo que simula o sistema da companhia fornecedora de água.\\


\section{Módulo WaterWatcher} \label{modww} %Water Watcher-----------------------------------------------
Este módulo é onde serão desenvolvidos a aplicação para os utilizadores e a lógica do servidor. Sendo assim, este módulo está dependente dos módulos WaterWatcherService e SimulCompany, sendo estas as únicas dependências entre os módulos do projeto.\\
Decidimos implementar a aplicação e o servidor no mesmo módulo dado que a plataforma nos permite implementar ações específicas do servidor ou do cliente num mesmo módulo, podendo distinguir quais os processos que irão ser executados nos servidores da aplicação ou no dispositivo dos clientes sem ser necessário criar mais estruturas de código.\\
Este módulo implementa toda a lógica e as vistas da aplicação para os utilizadores.

\section{Módulo WaterWatcherService} \label{modwws} %Water Watcher Service------------------------------
Este módulo é onde implementamos toda a lógica relativa ao armazenamento de informações que vamos utilizar neste sistema. No diagrama [ adicionar diagrama] podemos observar uma representação gráfica das várias entidades que armazenamos.\\
Os clientes têm os todos os atributos propostos na secção \ref{sec:dados} e têm ainda um atributo único que os identifica denominado idUser, que é gerado automaticamente aquando da criação de um novo utilizador. Este identificador é também o identificador de outra estrutura, que é a estrutura User. Esta estrutura contém as mesmas informações da estrutura Cliente, porém como esta é uma estrutura em qual a autenticação em aplicações construídas nesta plataforma se baseia \cite{outs:users} não poderíamos substituí-la.

\section{Módulo de Testes SimulCompany} \label{modsc}

\section{Ecrãs de Log in e Registo} \label{ecra:login}
A primeira página com a qual o utilizador interage é a página de log in. Esta página, para além de apresentar os contactos da empresa fornecedora, permite que o utilizador se registe ou se autentique com a sua conta já existente.
O registo no sistema segue o processo representado na figura \ref{fig:registo}.

\begin{figure}[h!]
\begin{center}
\resizebox{120mm}{!}{\includegraphics{diagramas/svg/Registo.jpg}}
\caption{Processo de registo no sistema.}
\label{fig:registo}
\end{center}
\end{figure}

Como descrito na figura, após o utilizador iniciar o registo e indicar o seu número de cliente, o sistema verifica se existe algum cliente registado na empresa fornecedora com esse número de cliente. Caso não exista nenhum cliente, é apresentada uma mensagem de erro. Se existir, o sistema verifica agora se este cliente já se encontra registado no serviço. Caso já se encontre registado, não será possível criar uma nova conta para este cliente, sendo assim apresentada uma nova mensagem de erro.\\
Sabendo agora que o cliente existe e não tem conta no sistema, é enviado um link para o email associado à conta do cliente no sistema da empresa fornecedora de água. Este link contém um código gerado aleatoriamente que permitirá ao utilizador finalizar o seu registo, definindo uma palavra-passe. Após este registo, o sistema obtém as informações deste cliente na empresa fornecedora e é então gerado um novo utilizador no sistema .\\
Para a geração deste código aleatório, recorremos à biblioteca RNGCryptoServiceProvider \cite{RNGCryptoServiceProvider} onde geramos um código de 20 caracteres (letras maiúsculas, minúsculas e números). Ao início deste código vai ser adicionado o número de cliente seguido por um ‘.’ , para que possamos identificar o cliente ao qual o código está associado apenas pelo código em si. Este código é depois apagado do sistema após o registo do cliente.\\
Depois de registado o utilizador poderá fazer o log in, indicando o seu número de cliente e a palavra-passe que definiu. 

\section{Ecrã de Informações e Estatísticas} \label{ecra:info}
No ecrã de informações e estatísticas o cliente pode consultar as suas contagens de forma gráfica e os detalhes das suas faturas para os seus vários contadores e em diferentes anos.\\
Caso esteja inscrito no programa de controlo semanal também poderá visualizar as suas várias contagens semanais de forma gráfica para os vários contadores, anos e meses.\\
Por não mantermos um registo das faturas mensais dos utilizadores na base de dados do sistema, estas são obtidas através de pedidos ao sistema informático da empresa fornecedora.\\
De forma a não fazer pedidos em excesso a esse sistema informático, quando o utilizador carrega a página, serão carregadas as suas faturas para o ano atual e o ano passado. Posteriormente, caso o utilizador selecione outro ano para obter as faturas, é feito um novo pedido à empresa, porém essas faturas ficam guardadas localmente para que não sejam feitos novos pedidos à empresa cada vez que o utilizador seleciona esse ano nessa mesma sessão de utilização. Ou seja, só é feito um pedido à empresa da primeira vez que o utilizador seleciona esse ano e contador nessa sessão.\\

\section{Ecrã de Definições} \label{ecra:def}
No ecrã de definições, são apresentadas ao cliente as informações relacionadas com a sua conta na empresa fornecedora de água, mais concretamente, o seu número de cliente, número de conta, nome, email e telefone.\\
O cliente poderá também alterar o seu email e telefone, sendo que esta informação será alterada neste sistema informático e no da empresa de água.\\
O cliente também poderá atualizar as suas informações, ou seja, caso tenha sido alterada alguma informação da sua conta na empresa, o cliente poderá atualizar a sua informação no sistema, sendo verificada e alterada a sua informação neste sistema informático, de acordo com a informação presente no sistema da empresa, de forma esta que fique consistente em ambos.

\section{Ecrã de Envio de Leituras} \label{ecra:leituras}
Esta página permite aos clientes do serviço enviar as suas contagens mensais e semanais, caso esteja inscrito no programa de controlo semanal.\\
O envio das contagens só é possível durante alguns dias, sendo eles o dia escolhido para o envio das contagens semanais e o dia seguinte, e os dias finais de cada mês para o envio das contagens mensais. Atualmente o dia definido para o início do envio das contagens mensais é o dia 25 de cada mês, porém este dia pode ser alterado.\\
Também é possível, nesta página, que o cliente se inscreva, ou desista do programa de controlo semanal de contagens. Para a inscrição apenas precisa de selecionar o dia de semana e hora a que pretende ser notificado e após finalizar a inscrição será relembrado semanalmente no tempo escolhido através de um email.\\
Por fim, o envio da contagem pode ser feito por texto, escrevendo a leitura atual do contador, ou através de uma fotografia, que encaminhará o utilizador para o ecrã de envio da fotografia.\\
Para ambos os casos, deverá ser selecionado o contador em questão, de entre os contadores apresentados numa caixa de seleção. Apenas são apresentados os contadores que ainda não têm contagem submetida no determinado mês ou semana.

\section{Ecrã de Envio de Fotografia do Contador} \label{ecra:foto}
Quando o utilizador inicia o processo de envio da fotografia do seu contador, a câmara do seu dispositivo é ativada e o ecrã passa a transmitir o que a câmara está a captar, aparecendo no centro do ecrã uma moldura retangular azul. O utilizador deve alinhar os algarismos que mostram a leitura (com fundo preto) com a moldura, como representado na figura \ref{fig:moldura}, dado que o sistema vai apenas recolher a imagem que está contida nessa moldura.\\
Posteriormente, a aplicação móvel envia a imagem para o servidor, local onde será feito o OCR.

\begin{figure}[h!]
\begin{center}
\resizebox{120mm}{!}{\includegraphics{diagramas/moldura.jpg}}
\caption{Esquematização do alinhamento da moldura com os caracteres da medição de água.}
\label{fig:moldura}
\end{center}
\end{figure}

Depois de o servidor efetuar o OCR, é enviado para a aplicação cliente o resultado desta operação e um número compreendido entre 0 e 100 que representa a confiança no resultado, ou seja, a probabilidade calculada pelo módulo de OCR de o resultado estar correto.\\
Caso a confiança seja menor de 75, é apresentado ao utilizador uma mensagem de erro de forma a que o utilizador envie uma nova fotografia.\\
Se for maior ou igual a 75, é apresentado ao utilizador o texto obtido para que este possa confirmar se este está correto e, nesse caso, submeter a sua leitura ou para que este possa enviar uma nova fotografia.\\

\section{Ecrã de Administração} \label{ecra:admin}
A página para os administradores permite a utilizadores com papeis de administração notificar utilizadores sob a forma de email, bem como apagar utilizadores da base de dados do sistema.\\
A notificação de utilizadores inicia-se quando o utilizador seleciona o destinatário da mensagem, que pode ser um utilizador em específico, identificado pelo seu número de cliente, todos os utilizadores registados cuja morada se encontre na freguesia indicada ou todos os utilizadores do sistema.\\
Como referido anteriormente, este utilizador também poderá apagar outros utilizadores da base de dados do sistema, apenas indicando o número de cliente do utilizador a apagar. Esta ação apaga as contagens semanais, os seus contadores e a sua informação pessoal na base de dados do sistema.




















